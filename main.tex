\documentclass[12pt]{article}
\usepackage{graphicx}
\usepackage{amsmath}
\usepackage{geometry}
\geometry{a4paper, margin=1in}

\title{Tic-Tac-Toe Game with Gesture Recognition Using YOLO}
\author{Ahmed Eslam\\
        Faculty of Engineering}
\date{\today}

\begin{document}

\maketitle

\begin{abstract}
    This document provides an overview of a Tic-Tac-Toe game implemented with gesture recognition using the YOLO model. The project involves setting up a gesture detection system, integrating it with a Tic-Tac-Toe game logic, and evaluating the system’s performance.
\end{abstract}

\section{Introduction}
\subsection{Background}
    Gesture recognition allows users to interact with applications through hand movements. This project implements a Tic-Tac-Toe game where players use hand gestures to make moves. The system uses YOLO (You Only Look Once), a popular object detection model, to identify gestures.

\subsection{Objectives}
    The primary objective is to develop a Tic-Tac-Toe game where moves are detected via hand gestures. The system should accurately recognize gestures and update the game board accordingly.

\section{Methodology}
\subsection{Data Collection and Model Training}
    For gesture recognition, we use a pre-trained YOLO model. The model was trained to recognize specific hand gestures representing X and O. The dataset used for training included various hand gesture images to ensure robustness in gesture detection.

\subsection{Game Implementation}
    \subsubsection{Tic-Tac-Toe Board Setup}
        The game board is a 3x3 grid where each cell can be empty, contain an X, or contain an O. The board is represented as a 2D NumPy array.

    \subsubsection{Gesture Detection}
        The YOLO model detects hand gestures in real-time. Each gesture corresponds to a specific class ID (e.g., 0 for X and 1 for O). The position of the hand gesture is mapped to the corresponding cell in the Tic-Tac-Toe grid.

    \subsubsection{Game Logic}
        Moves are processed based on detected gestures. The game checks for win conditions or a draw after each move. The game alternates between two players, updating the board and checking for game-ending conditions.

\section{Code Explanation}
\subsection{Drawing the Game Board}
    The function \texttt{draw\_board()} draws the Tic-Tac-Toe grid on the video feed using OpenCV. It divides the frame into a 3x3 grid with white lines.

\subsection{Placing X and O}
    The function \texttt{draw\_x\_o()} updates the game board display by drawing Xs and Os in the appropriate cells based on the current state of the board.

\subsection{Mapping Gesture to Grid}
    \texttt{map\_box\_to\_grid()} calculates which cell in the Tic-Tac-Toe grid corresponds to the center of the detected hand gesture bounding box.

\subsection{Gesture Detection}
    The function \texttt{detect\_gesture\_move()} uses the YOLO model to identify gestures. It returns the grid position based on the gesture detected (X or O).

\subsection{Win and Draw Conditions}
    \texttt{check\_win()} determines if the current player has won by checking rows, columns, and diagonals. \texttt{check\_draw()} checks if the board is full and no winner is found.

\subsection{Displaying Results}
    \texttt{display\_winner\_message()} shows a message indicating the game result (win or draw) and waits for a few seconds before closing the window.

\section{Testing and Evaluation}
    The system was tested by performing various gestures in front of the camera. The accuracy of gesture detection was evaluated, and the game logic was verified to ensure correct functionality.

\section{Conclusion}
    This project successfully implements a gesture-based Tic-Tac-Toe game using YOLO for real-time gesture recognition. The system provides an interactive and engaging way to play Tic-Tac-Toe through hand gestures.

\section{Future Work}
    Future improvements could include enhancing gesture recognition accuracy, expanding the range of gestures, and integrating more sophisticated game features.

\end{document}
